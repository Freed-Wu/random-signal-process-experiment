\documentclass[../main]{subfiles}
\begin{document}

\chapter{仿真}%
\label{cha:simulate}

\section{发射信号}%
\label{sec:emit}

\subsection{噪声}%
\label{sub:noise}

为了获取有确定信噪比的发射信号,我们要先获得噪声波形振幅的均值。根据假设%
~\ref{it:noise}和定义~\ref{def:noise},噪声$n(t)$\nomenclature{$n(t)$}{噪声}
分布见图~\ref{fig:noise_pdf}。

\begin{definition}[标准高斯复白噪声]%
  \label{def:noise}
  满足式~\ref{eq:noise}的信号$n(t)$。
\end{definition}

\begin{subnumcases}{\label{eq:noise}}%
  n(t) \sim N(0, 1)\\
  n(t) \in \mathbb{C}\\
  \mathrm{E}n(t)n(t + \tau) = \delta(\tau)
\end{subnumcases}

\begin{figure}[htbp]
  \centering
  \includegraphics[
    width = 0.8\linewidth,
  ]{fig/noise_pdf}
  \caption{噪声概率密度函数}%
  \label{fig:noise_pdf}
\end{figure}

\subsection{基带信号}%
\label{sub:baseband}

根据假设~\ref{it:hillbert}和定义~\ref{def:lfm},发射的基带波形
$s_0(t)$\nomenclature{$s_0(t)$}{发射的基带波形}见图~\ref{fig:baseband}。发射
的基带频谱$S_0(f)$\nomenclature{$S_0(f)$}{发射的基带频谱}见图%
~\ref{fig:baseband_spectrum}。

\begin{definition}[线性调频复信号]%
  \label{def:lfm}
  满足式~\ref{eq:baseband_wave}的信号$s_0(t)$。其中
  $\alpha$\nomenclature{$\alpha$}{角加频率}是角加频率,
  $k$\nomenclature{$k$}{加频率}是加频率,$\tau$\nomenclature{$\tau$}{时宽}是时
  宽,$B$\nomenclature{$B$}{带宽}是带宽,
  $T_\mathrm{r}$\nomenclature{$T_\mathrm{r}$}{脉冲重复周期}是脉冲重复周期,
  $A$\nomenclature{$A$}{振幅}是根据式~\ref{eq:a}得到的振幅。
\end{definition}

\begin{align}
  \label{eq:baseband_wave}
  s_0(t) = & A\exp{\jmath\frac{\alpha t^2}{2}}
  \mathrm{rect}\left(\frac{t\%T_\mathrm{r}}{\tau} + \frac{1}{2}\right)\\
  \alpha = & 2\pi k\\
  k = & \frac{B}{\tau}
\end{align}

\begin{figure}[htbp]
  \centering
  \begin{subfigure}[htbp]{0.45\linewidth}
    \centering
    \includegraphics[
      width = \linewidth,
    ]{fig/baseband_wave}
    \caption{基带波形}%
    \label{fig:baseband_wave}
  \end{subfigure}
  \quad
  \begin{subfigure}[htbp]{0.45\linewidth}
    \centering
    \includegraphics[
      width = \linewidth,
    ]{fig/baseband_wave_band}
    \caption{基带波形通带}%
    \label{fig:baseband_wave_band}
  \end{subfigure}
  \caption{基带信号}%
  \label{fig:baseband}
\end{figure}

\begin{figure}[htbp]
  \centering
  \includegraphics[
    width = 0.8\linewidth,
  ]{fig/baseband_spectrum}
  \caption{基带频谱}%
  \label{fig:baseband_spectrum}
\end{figure}

根据假设~\ref{it:doppler},~\ref{it:time_switch}和定理%
~\ref{th:max_r},~\ref{th:max_v},~\ref{th:min_r},~\ref{th:min_v},
问题~\ref{ex:1}可解。

\begin{theorem}[不模糊距离]%
  \label{th:max_r}
  不模糊且可分辨的最大距离$\max{R}$\nomenclature{$R$}{雷达可测距离}由
  式~\ref{eq:max_r}决定。其中$c$\nomenclature{$c$}{光速}是光速。
\end{theorem}

\begin{theorem}[不模糊速度]%
  \label{th:max_v}
  不模糊且可分辨的最大速度由$\max{v}$\nomenclature{$v$}{雷达可测速度}
  式~\ref{eq:max_v}决定。其中$\lambda$\nomenclature{$\lambda$}{波长}是波长,
  $f_\mathrm{r}$\nomenclature{$f_\mathrm{r}$}{脉冲重复频率}是脉冲重复频率。
\end{theorem}

\begin{theorem}[距离分辨力]%
  \label{th:min_r}
  不模糊且可分辨的最大距离由式~\ref{eq:min_r}决定。
\end{theorem}

\begin{theorem}[速度分辨力]%
  \label{th:min_v}
  不模糊且可分辨的最大速度由式~\ref{eq:min_v}决定。其中
  $f_\mathrm{d}$\nomenclature{$f_\mathrm{d}$}{相干处理频率}是相干处理频率。
  $t_\mathrm{d}$\nomenclature{$t_\mathrm{d}$}{相干处理时间}是相干处理时间。
\end{theorem}

\begin{align}
  \label{eq:max_r}
  \max{R} = & \frac{cT_\mathrm{r}}{2}\\
  \label{eq:max_v}
  \max{v} = & \frac{\lambda f_\mathrm{r}\varparallel 2c}{2}\\
  \label{eq:min_r}
  \min{R} = & \frac{c}{2B}\\
  \label{eq:min_v}
  \min{v} = & \frac{\lambda f_\mathrm{d}\varparallel 2c}{2}\\
  f_\mathrm{r} = & \frac{1}{T_\mathrm{r}}\\
  \lambda = & \frac{c}{f_\mathrm{c}}\\
  f_\mathrm{d} = & \frac{1}{T_\mathrm{d}}
\end{align}

\begin{Answer}[ref = ex:1]
  不模糊距离、不模糊速度、距离分辨力、速度分辨力见表~\ref{tab:parameter}。
\end{Answer}

\begin{table}[htbp]
  \centering
  \caption{参数}%
  \label{tab:parameter}
  \csvautobooktabular{tab/parameter.csv}
\end{table}

\section{自相关函数}%
\label{sec:self_coherent}

\subsection{发射的基带脉冲压缩信号}%
\label{sub:self_coherent}

根据定理~\ref{th:match_filter}匹配滤波器冲激响应波形
$h(t)$\nomenclature{$h(t)$}{匹配滤波器冲激响应}见图~\ref{fig:mf_wave}。

\begin{theorem}[匹配滤波器]%
  \label{th:match_filter}
  能够使输出信噪比最大的滤波器冲激响应波形$h(t)$满足式\ref{eq:match_filter}。
\end{theorem}

\begin{align}
  \label{eq:match_filter}
  h(t) = s_0^*(\frac{T + \tau}{2} - t)
\end{align}

\begin{figure}[htbp]
  \centering
  \includegraphics[
    width = 0.8\linewidth,
  ]{fig/mf_wave}
  \caption{匹配滤波器冲激响应波形}%
  \label{fig:mf_wave}
\end{figure}

发射的基带自相关函数$R_{s_0}(t)$\nomenclature{$R_{s_0}(t)$}{发射的基带自相关
函数}是非因果信号。但根据定理~\ref{th:self_coherent}和
定义~\ref{def:pulse_compression},可以用发射的基带脉冲压缩信号代替基带自相关
函数$R_{s_0}(t)$。发射的基带脉压波形$s_0^0(t)$\nomenclature{$s_0^0(t)$}{发射
的基带脉压波形}见图~\ref{fig:self_coherent}。

\begin{definition}[脉冲压缩信号]%
  \label{def:pulse_compression}
  满足式~\ref{eq:pulse_compression}的信号$s^0(t)$。
\end{definition}

\begin{align}
  \label{eq:pulse_compression}
  s^0(t) = s(t) * h(t)
\end{align}

\begin{theorem}[发射的基带自相关函数与发射的基带脉冲压缩信号的关系]%
  \label{th:self_coherent}
  根据式~\ref{eq:self_coherent_wave},发射的基带自相关函数与发射的基带脉冲压
  缩信号的波形相同。
\end{theorem}

\begin{align}
  \label{eq:self_coherent_wave}
  R_{s_0}(t) = & s_0(t) * s_0^*(-t)\\
  s_0^0(t) = & s_0(t) * h(t)
\end{align}

\begin{figure}[htbp]
  \centering
  \begin{subfigure}[htbp]{0.45\linewidth}
    \centering
    \includegraphics[
      width = \linewidth,
    ]{fig/self_coherent_wave}
    \caption{发射的基带脉压波形}%
    \label{fig:self_coherent_wave}
  \end{subfigure}
  \quad
  \begin{subfigure}[htbp]{0.45\linewidth}
    \centering
    \includegraphics[
      width = \linewidth,
    ]{fig/self_coherent_band}
    \caption{发射的基带信号的脉压波形通带}%
    \label{fig:self_coherent_band}
  \end{subfigure}
  \caption{发射的基带脉压信号}%
  \label{fig:self_coherent}
\end{figure}

发射的基带脉压波形波特图见图~\ref{fig:self_coherent_db}。将采样频率增大到最低
不失真采样频率的10倍,发射的基带脉压波形通带局部放大波特图见
图~\ref{fig:self_coherent_enlargement_hf_db}。

\begin{figure}[htbp]
  \centering
  \begin{subfigure}[htbp]{0.45\linewidth}
    \centering
    \includegraphics[
      width = \linewidth,
    ]{fig/self_coherent_band_db}
    \caption{发射的基带脉压波形通带波特图}%
    \label{fig:self_coherent_band_db}
  \end{subfigure}
  \quad
  \begin{subfigure}[htbp]{0.45\linewidth}
    \centering
    \includegraphics[
      width = \linewidth,
    ]{fig/self_coherent_enlargement_db}
    \caption{发射的基带脉压波形局部放大波特图}%
    \label{fig:self_coherent_enlargement_db}
  \end{subfigure}
  \caption{发射的基带脉压波形波特图}%
  \label{fig:self_coherent_db}
\end{figure}

\begin{figure}[htbp]
  \centering
  \includegraphics[
    width = 0.8\linewidth,
  ]{fig/self_coherent_enlargement_hf_db}
  \caption{发射的基带脉压波形高采样率局部放大波特图}%
  \label{fig:self_coherent_enlargement_hf_db}
\end{figure}

\subsection{发射的基带脉冲压缩信号加窗}%
\label{sub:self_coherent_window}

发射的基带脉压波形加窗高采样率波特图见图~\ref{fig:self_coherent_window_hf_db}
。

\begin{figure}[htbp]
  \centering
  \begin{subfigure}[htbp]{0.45\linewidth}
    \centering
    \includegraphics[
      width = \linewidth,
    ]{fig/self_coherent_window_enlargement_hf_db}
    \caption{自相关函数加窗高采样率局部放大波特图}%
    \label{fig:self_coherent_window_enlargement_hf_db}
  \end{subfigure}
  \quad
  \begin{subfigure}[htbp]{0.45\linewidth}
    \centering
    \includegraphics[
      width = \linewidth,
    ]{fig/self_coherent_window_fft_hf_db}
    \caption{自相关函数加窗高采样率频谱波特图}%
    \label{fig:self_coherent_window_fft_hf_db}
  \end{subfigure}
  \caption{自相关函数加窗高采样率波特图}%
  \label{fig:self_coherent_window_hf_db}
\end{figure}

\begin{Answer}[ref = ex:2]
  不同窗函数时的第一旁瓣高度、\SI{4}{\dB}输出脉冲宽度、主瓣展宽的倍数见
  表~\ref{tab:self_coherent_window}。
\end{Answer}

\begin{table}[htbp]
  \centering
  \caption{不同窗函数抑制旁瓣时的旁瓣大小和主瓣展宽的倍数}%
  \label{tab:self_coherent_window}
  \scriptsize
  \csvautobooktabular{tab/self_coherent_window.csv}
\end{table}

\section{单目标动目标检测}%
\label{sec:single_mtd}

\subsection{回波信号}%
\label{sub:echo}

根据式\ref{th:echo_wave},回波波形$s_1(t)$\nomenclature{$s_1(t)$}{回波波形}、
频谱$S_1(t)$\nomenclature{$S_1(f)$}{回波频谱}见图~\ref{fig:echo}。

\begin{theorem}[多普勒效应]%
  \label{th:echo_wave}
  回波波形$s_1(t)$满足式~\ref{eq:echo_wave},其中
  $f_\mathrm{v}(v)$\nomenclature{$f_\mathrm{v}(v)$}{多普勒频移}为多普勒频移。
  取以雷达为原点的矢径$R$正方向为速度$v$正方向。取频率增加为多普勒频移
  $f_\mathrm{v}(v)$的正方向。
\end{theorem}

\begin{align}
  \label{eq:echo_wave}
  s_1(t) = & s_0(t)\exp{\jmath 2\pi f_\mathrm{v}(v)t}\\
  f_\mathrm{v}(v) = & -\frac{2v}{\lambda}\frac{c}{c - v}
\end{align}

\begin{figure}[htbp]
  \centering
  \begin{subfigure}[htbp]{0.45\linewidth}
    \centering
    \includegraphics[
      width = \linewidth,
    ]{fig/echo_wave}
    \caption{回波波形}%
    \label{fig:echo_wave}
  \end{subfigure}
  \quad
  \begin{subfigure}[htbp]{0.45\linewidth}
    \centering
    \includegraphics[
      width = \linewidth,
    ]{fig/echo_spectrum}
    \caption{回波频谱}%
  \end{subfigure}
  \caption{回波信号}%
  \label{fig:echo}
\end{figure}

脉压波形$s_1^0(t)$\nomenclature{$s_1^0(t)$}{脉压波形}见图%
~\ref{fig:pulse_compression_wave}。

\begin{figure}[htbp]
  \centering
  \includegraphics[
    width = 0.8\linewidth,
  ]{fig/pulse_compression_wave}
  \caption{脉压波形}%
  \label{fig:pulse_compression_wave}
\end{figure}

\subsection{动目标检测}%
\label{sub:mtd}

根据定义~\ref{def:mtd},动目标检测波形$s_1^0(t, n)$\nomenclature{$s_1^0(t,
n)$}{动目标检测波形}及其慢时间频谱$S_1^0(t, F)$\nomenclature{$S_1(t, F)$}{动
目标检测慢时间频谱}见图~\ref{fig:mtd}。

\begin{definition}[动目标检测信号]%
  \label{def:mtd}
  满足式~\ref{eq:mtd}的信号$s_1^0(t, n)$,其中$t$是快时间,$n$是慢时间。
\end{definition}

\begin{align}
  \label{eq:mtd}
  s_1^0(t, n) = & s_1^0(t + nT_\mathrm{r})\\
  t \in & \left[0, T_\mathrm{r}\right)\\
  n \in & \mathbb{N}
\end{align}

\begin{figure}[htbp]
  \centering
  \begin{subfigure}[htbp]{0.45\linewidth}
    \centering
    \includegraphics[
      width = \linewidth,
    ]{fig/mtd_wave}
    \caption{动目标检测波形}%
    \label{fig:mtd_wave}
  \end{subfigure}
  \quad
  \begin{subfigure}[htbp]{0.45\linewidth}
    \centering
    \includegraphics[
      width = \linewidth,
    ]{fig/mtd_slow_spectrum}
    \caption{动目标检测慢时间频谱}%
    \label{fig:mtd_slow_spectrum}
  \end{subfigure}
  \caption{动目标检测}%
  \label{fig:mtd}
\end{figure}

动目标检测慢时间频谱慢时间视图$\max\limits_{t}S_1^0(t, F)$及其波特图见
图~\ref{fig:mtd_slow_spectrum_slow_view_and_db}。

\begin{figure}[htbp]
  \centering
  \begin{subfigure}[htbp]{0.45\linewidth}
    \centering
    \includegraphics[
      width = \linewidth,
    ]{fig/mtd_slow_spectrum_slow_view}
    \caption{动目标检测慢时间频谱慢时间视图}%
    \label{fig:mtd_slow_spectrum_slow_view}
  \end{subfigure}
  \quad
  \begin{subfigure}[htbp]{0.45\linewidth}
    \centering
    \includegraphics[
      width = \linewidth,
    ]{fig/mtd_slow_spectrum_slow_view_db}
    \caption{动目标检测慢时间频谱慢时间视图波特图}%
    \label{fig:mtd_slow_spectrum_slow_view_db}
  \end{subfigure}
  \caption{动目标检测慢时间频谱慢时间视图及其波特图}%
  \label{fig:mtd_slow_spectrum_slow_view_and_db}
\end{figure}

动目标检测加窗慢时间频谱慢时间视图波特图$\max\limits_{t}S_1^0(t, F)$见
图~\ref{fig:mtd_slow_spectrum_slow_view_and_db}。

\begin{figure}[htbp]
  \centering
  \includegraphics[
    width = 0.8\linewidth,
  ]{fig/mtd_window_slow_spectrum_slow_view_db}
  \caption{动目标检测加窗慢时间频谱慢时间视图波特图}%
  \label{fig:mtd_window_slow_spectrum_slow_view_db}
\end{figure}

\subsection{多普勒敏感}%
\label{sub:doppler_sensitivity}

根据定义~\ref{def:doppler_sensitivity},脉压波形与速度的关系波特图见
图~\ref{fig:pulse_compression_wave_velocity_db}。脉压波形与速度的关系剖视图波
特图见图~\ref{fig:doppler_sensitivity}。

\begin{definition}[多普勒敏感]%
  \label{def:doppler_sensitivity}
  脉压波形与速度的关系$s_1^0(t, v)$。
\end{definition}

\begin{figure}[htbp]
  \centering
  \includegraphics[
    width = 0.8\linewidth,
  ]{fig/pulse_compression_wave_velocity_db}
  \caption{脉压波形与速度的关系波特图}%
  \label{fig:pulse_compression_wave_velocity_db}
\end{figure}

\begin{figure}[htbp]
  \centering
  \begin{subfigure}[htbp]{0.45\linewidth}
    \centering
    \includegraphics[
      width = \linewidth,
    ]{fig/pulse_compression_wave_velocity_section_view_time_db}
    \caption{脉压波形与速度的关系时间剖视图波特图}%
    \label{fig:pulse_compression_wave_velocity_section_view_time_db}
  \end{subfigure}
  \quad
  \begin{subfigure}[htbp]{0.45\linewidth}
    \centering
    \includegraphics[
      width = \linewidth,
    ]{fig/pulse_compression_wave_velocity_section_view_time_enlargement_db}
    \caption{脉压波形与速度的关系时间剖视图波特图局部放大图}%
    \label{fig:pulse_compression_wave_velocity_section_view_time_enlargement_db}
  \end{subfigure}
    \caption{多普勒敏感}%
    \label{fig:doppler_sensitivity}
\end{figure}

根据定义~\ref{def:doppler_tolerance},只要给出脉压增益最低许可容限即可求出多
普勒容限。

\begin{definition}[多普勒容限]%
  \label{def:doppler_tolerance}
  脉压增益降低到最低许可容限时的多普勒频移。
\end{definition}

\begin{Answer}[ref = ex:3]
  脉冲压缩信噪比增益、 FFT 信噪比增益、时宽、带宽见表~\ref{tab:measure}。多卜
  勒敏感现象及其性能损失见图~\ref{fig:doppler_sensitivity}。
\end{Answer}

\begin{table}[htbp]
  \centering
  \caption{测量结果}%
  \label{tab:measure}
  \csvautobooktabular{tab/measure.csv}
\end{table}

\section{双目标动目标检测}%
\label{sec:double_mtd}

\begin{Answer}[ref = ex:4]
  \begin{itemize}
    \item 大目标旁瓣掩盖小目标见图~\ref{fig:amplitude_distinguish_db};
    \item 距离、速度分辨的临界情况见图~\ref{fig:distinguish}。
  \end{itemize}
\end{Answer}

\begin{figure}[htbp]
  \centering
  \begin{subfigure}[htbp]{0.45\linewidth}
    \centering
    \includegraphics[
    width = \linewidth,
    ]{fig/amplitude_distance_distinguish_db}
    \caption{大目标旁瓣掩盖小目标(距离)}%
    \label{fig:amplitude_distance_distinguish_db}
  \end{subfigure}
  \quad
  \begin{subfigure}[htbp]{0.45\linewidth}
    \centering
    \includegraphics[
      width = \linewidth,
    ]{fig/amplitude_velocity_distinguish_db}
    \caption{大目标旁瓣掩盖小目标(速度)}%
    \label{fig:amplitude_velocity_distinguish_db}
  \end{subfigure}
  \caption{大目标旁瓣掩盖小目标}%
  \label{fig:amplitude_distinguish_db}
\end{figure}

\begin{figure}[htbp]
  \centering
  \begin{subfigure}[htbp]{0.45\linewidth}
    \centering
    \includegraphics[
      width = \linewidth,
    ]{fig/distance_distinguish_db}
    \caption{距离分辨临界情况}%
    \label{fig:distance_distinguish_db}
  \end{subfigure}
  \quad
  \begin{subfigure}[htbp]{0.45\linewidth}
    \centering
    \includegraphics[
      width = \linewidth,
    ]{fig/velocity_distinguish}
    \caption{速度分辨临界情况}%
    \label{fig:velocity_distinguish}
  \end{subfigure}
  \caption{分辨临界情况}%
  \label{fig:distinguish}
\end{figure}

\end{document}

