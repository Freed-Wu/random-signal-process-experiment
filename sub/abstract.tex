\documentclass[../main]{subfiles}
\begin{document}

\begin{abstract}
  线性调频 (LFM)技术在雷达、声纳技术中有广泛应用。本文针对题目要求,使
  用\href{https://github.com/JuliaLang/julia}{Julia}设计了一个参数可调的线性
  调频脉冲雷达仿真系统。\cite{FFTW.jl-2005}

  针对问题~\ref{ex:1},计算得到该雷达的不模糊距离和速度,距离和速度分辨率;

  针对问题~\ref{ex:2},仿真了 LFM 信号的自相关函数,并对脉压信号进行了加窗,
  比较了不同窗函数对旁瓣高度,主瓣宽度,\SI{4}{\dB}脉冲宽度的影响;

  针对问题~\ref{ex:3},仿真了单目标时的脉压,动目标检测,多普勒敏感的情况。将
  实验的结果与理论比较,并计算了误差;

  针对问题~\ref{ex:4},仿真出大目标旁瓣盖掩盖小目标、距离和速度刚好
  无法分辨的情况。

  最后列出了程序运行的时间和对内存的消耗等指标。此外,本程序具有2个特点:

  \begin{description}
    \item[全自动]只需运行此程序即会自动得到结果,无需截图,无需
      复制粘贴,程序会自己新建文件夹,自己计算结果,并将运行后得到的图片和表
      格保存到文件夹下。完全无需人手动操作;
    \item[参数可调]真正意义的参数可调。程序主体采用函数式编程,除了一开
      始输入的参数外,不依赖任何外来参数。所以只修改一开始输入的参数,即可
      正确运行。
  \end{description} 
  \begin{keyword}
    线性调频脉冲雷达,动目标检测,多普勒敏感
  \end{keyword}
\end{abstract}

\begin{abstract*}
  LFM technology is widely used in radar and sonar technology. In this paper,
  a simulation system of LFM pulse radar with adjustable parameters is
  designed by
  \href{https://github.com/JuliaLang/julia}{Julia}.\cite{FFTW.jl-2005}

  In order to solve Problem~\ref{ex:1}, the range and velocity resolution are calculated;

  Aiming at Problem~\ref{ex:2}, the autocorrelation function of LFM signal is
  simulated, and the pulse pressure signal is windowed. The effects of
  different window functions on the side lobe height, main lobe width, and
  the pulse width of \SI{4}{\dB} are compared;

  Aiming at Problem~\ref{ex:3}, the pulse pressure, moving target detection
  and Doppler sensitivity of single target are simulated. The experimental
  results are compared with the theoretical results, and the errors are
  calculated;

  In order to solve Problem~\ref{ex:4}, it is simulated that the sidelobe
  cover of a large target can cover a small target, and the distance and
  speed are just indistinguishable.

  Finally, the running time of the program and the consumption of memory are
  listed. In addition, the program has two characteristics:

  \begin{description}
    \item[fully automatic]Just run this program to get the results
      automatically, without screenshots, copying and pasting. The program
      will create a new folder by itself, calculate the results by itself,
      and save the pictures and tables obtained after running to the folder.
      No manual operation is required at all;
    \item[parameter adjustable] the real parameter is adjustable. The main
      body of the program adopts functional programming, which does not
      depend on any foreign parameters except the parameters input at the
      beginning. Therefore, only the parameters entered at the beginning can
      be modified to run correctly.
  \end{description} 
  \begin{keyword*}
    LFM Pulse Radar, Moving target detection, Doppler sensitive
  \end{keyword*}
\end{abstract*}

\end{document}

